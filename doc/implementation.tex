\chapter{Технологический раздел}

\section{Обзор СУБД с построчным хранением}

В данном подразделе будут рассмотрены популярные построчные СУБД, которые могут быть использованы для реализации хранения в разрабатываемом программном продукте.\\

\noindent\textbf{PostgreSQL}

PostgreSQL \cite{postgresql} -- это свободно распространяемая объектно-реляционная система управления базами данных, наиболее развитая из открытых СУБД в мире и являющаяся реальной альтернативой коммерческим базам данных \cite{postgresql-fact}.

PostgreSQL предоставляет транзакции со свойствами атомарности, согласованности, изоляции, долговечности (ACID \cite{acid}), автоматически обновляемые представления, материализованные представления, триггеры, внешние ключи и хранимые процедуры. Данная СУБД предназначена для обработки ряда рабочих нагрузок, от отдельных компьютеров до хранилищ данных или веб-сервисов с множеством одновременных пользователей. 

Рассматриваемая СУБД управляет параллелизмом с помощью технологии управления многоверсионным параллелизмом (англ. MVCC \cite{mvcc}). Эта технология дает каждой транзакции <<снимок>> текущего состояния базы данных, позволяя вносить изменения, не затрагивая другие транзакции. Это в значительной степени устраняет необходимость в блокировках чтения (англ. read lock \cite{r-lock}) и гарантирует, что база данных поддерживает принципы ACID. \\

\noindent\textbf{Oracle Database}

Oracle Database \cite{oracle} -- объектно-реляционная система управления базами данных компании Oracle \cite{oracle-company}. На данный момент, рассматриваемая СУБД является самой популярной в мире. \cite{oracle-popular}

Все транзакции Oracle Database соответствуют обладают свойствами ACID, поддерживает триггеры, внешние ключи и хранимые процедуры. Данная СУБД подходит для разнообразных рабочих нагрузок и может использоваться практически в любых задачах. Особенностью Oracle Database является быстрая работа с большими массивами данных.

Oracle Database может использовать один или более методов параллелизма. Сюда входят механизмы блокировки для гарантии монопольного использования таблицы одной транзакцией, методы временных меток, которые разрешают сериализацию транзакций и планирование транзакций на основе проверки достоверности. \\

\noindent\textbf{MySQL}

MySQL \cite{mysql} -- свободная реляционная система управления базами данных. Разработку и поддержку MySQL осуществляет корпорация Oracle.

Рассматриваемая СУБД имеет два основных движка хранения данных: InnoDB \cite{innodb} и myISAM \cite{myisam}. Движок InnoDB полностью полностью совместим с принципами ACID, в отличии от движка myISAM. СУБД MySQL подходит  для использования при разработке веб-приложений, что объясняется очень тесной интеграцией с популярными языками PHP \cite{php} и Perl \cite{perl}.

Реализация параллелизма в СУБД MySQL реализовано с помощью механизма блокировок, который обеспечивает одновременный доступ к данным. \\

\noindent\textbf{Вывод}

В результате анализа для решения задачи выбрана СУБД PostgreSQL.