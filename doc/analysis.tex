\chapter{Аналитический раздел}
В данном разделе будут рассмотрены существующие решения, формализована решаемая задача, выбрана СУБД для ее решения. 

\section{Обзор существующих решений}
В 2020 году WWF России представил проект, который призван помочь жителям нашей страны оперативно сообщать об авариях и инцидентах \cite{wwf1}. Идея была размещена в рамках форума "Сильные идеи для нового времени" \cite{forum}. С помощью специальной формы на сайте каждый сможет оперативно передать фото с места аварии, координаты инцидента и снабдить свое сообщение комментарием. По информации с места событий будут запрашиваться оперативные данные космического мониторинга. Однако на момент написания курсовой работы нет сведений о судьбе данного проекта.

WWF России и Fairy, бренд компании Procter\&Gamble, запустили национальную программу общественного мониторинга аварийных экологических ситуаций. Программа позволяет своевременно отслеживать и оповещать дежурные службы об инцидентах, аналогичных катастрофам на побережье Авачинской бухты на Камчатке, разливе в Норильске или аварии на продуктопроводе на реке Оби \cite{wwf2}. В рамках данного проекта у пользователей есть возможность получить информацию об инцидентах в с помощью интерактивной карты. Однако пользователи не могут загружать информацию об инцидентах в систему. Также данная программа узконаправлена -- данные собираются только об инцидентах в нефтегазовом секторе. 

\section{Формализация задачи}
Под экологическим инцидентом будем понимать любое происшествие, которое привело или может привести к неблагоприятным последствиям для окружающей среды. 

Разрабатываемая система должна выделять следующие типы экологических инцидентов: 
\begin{itemize}
	\item разлив нефти или нефтепродуктов;
	\item выброс радиоактивных веществ;
	\item выброс аварийно химически опасных веществ;
	\item выброс биологически опасных веществ;
	\item пожар;
	\item несанкционированных свалка, скопление мусора;
	\item другие экологические инциденты.
\end{itemize}

Каждая запись об экологическом инциденте должна содержать следующие данные:
\begin{itemize}
	\item краткое описание;
	\item тип;
	\item координаты;
	\item дата;
	\item статус (подтвержден/не подтвержден);
	\item пользователь, опубликовавший инцидент.
\end{itemize}

Каждая запись об экологическом инциденте может содержать следующие данные:
\begin{itemize}
	\item фотография;
	\item комментарий.
\end{itemize}

Работа пользователей с базой экологических инцидентов должна осуществляться посредством клиент-серверного веб-приложения с возможностью авторизации. Приложение должно поддерживать работу четырех типов пользователей со следующими возможностями: 
\begin{itemize}
	\item неавторизованный пользователь:
	\begin{itemize}
		\item просмотр записей об экологических инцидентах в виде списка;
		\item просмотр записей об экологических инцидентах в виде карты;
		\item регистрация нового аккаунта или вход в существующий.
	\end{itemize}
	\item авторизованный пользователь:
	\begin{itemize}
		\item все возможности неавторизованного пользователя;
		\item добавление записи об экологическом инциденте;
		\item удаление собственной записи об инциденте со статусом "неподтвержден";
		\item редактирование собственной записи об инциденте со статусом "неподтвержден";
		\item запрос на удаление собственной записи об инциденте со статусом "подтвержден";
		\item запрос удаление собственной записи об инциденте со статусом "подтвержден";
		\item выход из аккаунта.
	\end{itemize}
	\item модератор:
	\begin{itemize}
		\item все возможности авторизованного пользователя;
		\item установка статуса инцидента;
		\item установка для инцидента метки "Экологическая катастрофа";
		\item принятие/отклонение запроса авторизованного пользователя на редактирование записи об инциденте со статусом "подтвержден";
		\item принятие/отклонение запроса авторизованного пользователя на удаление записи об инциденте со статусом "подтвержден";
		\item удаление записи об инциденте;
		\item редактирование записи об инциденте.
	\end{itemize}
	\item администратор:
	\begin{itemize}
		\item все возможности модератора;
		\item назначение авторизованному пользователю роли модератора;
		\item снятие авторизованного пользователя с роли модератора. 
	\end{itemize}
\end{itemize}

\section{Анализ баз данных по способу хранения}

Базы данных, по способу хранения, делятся на две группы -- строковые и колоночные. Каждый из этих типов служит для выполнения для определенного рода задач. \\

\noindent\textbf{Строковые базы данных}

Строковыми базами даных называются такие базы данных, записи которых в памяти представляются построчно. Строковые базы данных используются в транзакционных системах (англ. OLTP \cite{OLTP}). Для таких систем характерно большое количество коротких транзакций с операциями вставки, обновления и удаления данных - \texttt{INSERT}, \texttt{UPDATE}, \texttt{DELETE}. 

Основной упор в системах OLTP делается на очень быструю обработку запросов, поддержание целостности данных в средах с множественным доступом и эффективность, которая измеряется количеством транзакций в секунду. 

Схемой, используемой для хранения транзакционных баз данных, является модель сущностей, которая включает в себя запросы, обращающиеся к отдельным записям. Так же, в OLTP-системах есть подробные и текущие данных.\\

\noindent\textbf{Колоночные базы данных}

Колоночными базами данных называются базы данных, записи которых в памяти представляются по столбцам. Колоночные базы данных используется в аналитических системах (англ. OLAP \cite{olap}). OLAP характеризуется низким объемом транзакций, а запросы часто сложны и включают в себя агрегацию. Время отклика для таких систем является мерой эффективности.

OLAP-системы широко используются методами интеллектуального анализа данных. В таких базах есть агрегированные, исторические данные, хранящиеся в многомерных схемах. 

\section{Вывод}
В данном разделе была формализована решаемая задача, проведен обзор существующих аналогов и СУБД. Для решения задачи выбран построчный способ хранения данных, так как:

\begin{itemize}
	\item задача предполагает постоянное добавление и изменение данных;
	\item задача предполагает быструю отзывчивость на запросы пользователя;
	\item задача не предполагает выполнения аналитических запросов.
\end{itemize}

