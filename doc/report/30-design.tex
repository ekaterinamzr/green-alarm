\chapter{Конструкторская часть}
В данном разделе будет спроектирована базы данных.

\section{Проектирование базы данных}
Для основной логики приложения была спроектирована база данных, представленная в виде ER-модели в нотации Мартина.  

\imgw{er}{h!}{1.0\textwidth}{ER-модель в нотации Мартина}

База данных будет состоять из следующих сущностей:
\begin{enumerate}
\item Users -- пользователи;
\item Roles -- роли пользователя;
\item Incidents -- экологические инциденты;
\item Statuses -- статусы экологических инцидентов;
\item Types -- типы экологических инцидентов.
\end{enumerate}

\newpage
\noindent\textbf{Сущность Users}

Сущность Users содержит информацию о пользователях:
\begin{table}[!ht]
    \caption{Описание полей таблицы \texttt{Users}}
    \label{tbl:users}
    \begin{center}
        \begin{tabular}{|p{0.2\textwidth}p{0.6\textwidth}|}
            \hline
            \textbf{Поле} & \textbf{Значение} \\\hline
            \texttt{id} & Уникальный идентификатор \\\hline
            \texttt{first\_name} & Имя\\\hline
            \texttt{last\_name} & Фамилия \\\hline
            \texttt{username} & Псевдоним \\\hline
            \texttt{email} & Адресс электронной почты \\\hline
            \texttt{user\_password} & Хешированный пароль \\\hline
            \texttt{user\_role} & Роль \\\hline
        \end{tabular}
    \end{center}
\end{table}

\noindent\textbf{Сущность Incidents}

Сущность Incidents содержит информацию об экологических инцидентах:
\begin{table}[!ht]
    \caption{Описание полей таблицы \texttt{Incidents}}
    \label{tbl:incidents}
    \begin{center}
        \begin{tabular}{|p{0.3\textwidth}p{0.6\textwidth}|}
            \hline
            \textbf{Поле} & \textbf{Значение} \\\hline
            \texttt{id} & Уникальный идентификатор \\\hline
            \texttt{incident\_name} & Название (краткое описание) \\\hline
            \texttt{incident\_date} & Дата\\\hline
            \texttt{country} & Страна \\\hline
            \texttt{latitude} & Широта \\\hline
            \texttt{longitude} & Долгота \\\hline
            \texttt{publication\_date} & Дата публикации \\\hline
            \texttt{comment} & Комментарий\\\hline
            \texttt{incident\_status} & Статус \\\hline
            \texttt{incident\_type} & Тип \\\hline
            \texttt{author} & Пользователь, опубликовавший запись об инциденте \\\hline
        \end{tabular}
    \end{center}
\end{table}

\newpage

\noindent\textbf{Сущность Roles}

Сущность Roles содержит информацию о существующих в системе ролях пользователй:
\begin{table}[!ht]
	\caption{Описание полей таблицы \texttt{Roles}}
	\label{tbl:roles}
	\begin{center}
		\begin{tabular}{|p{0.2\textwidth}p{0.6\textwidth}|}
			\hline
			\textbf{Поле} & \textbf{Значение} \\\hline
			\texttt{id} & Уникальный идентификатор. \\\hline
			\texttt{role\_name} & Название. \\\hline
		\end{tabular}
	\end{center}
\end{table}

\noindent\textbf{Сущность Statuses}

Сущность Statuses содержит информацию о существующих в системе статусах инцидентов:
\begin{table}[!ht]
	\caption{Описание полей таблицы \texttt{Statuses}}
	\label{tbl:statuses}
	\begin{center}
		\begin{tabular}{|p{0.2\textwidth}p{0.6\textwidth}|}
			\hline
			\textbf{Поле} & \textbf{Значение} \\\hline
			\texttt{id} & Уникальный идентификатор \\\hline
			\texttt{role\_name} & Название \\\hline
		\end{tabular}
	\end{center}
\end{table}

\noindent\textbf{Сущность Types}

Сущность Types содержит информацию о существующих в системе типах инцидентов:
\begin{table}[!ht]
	\caption{Описание полей таблицы \texttt{Types}}
	\label{tbl:types}
	\begin{center}
		\begin{tabular}{|p{0.2\textwidth}p{0.6\textwidth}|}
			\hline
			\textbf{Поле} & \textbf{Значение} \\\hline
			\texttt{id} & Уникальный идентификатор \\\hline
			\texttt{role\_name} & Название \\\hline
		\end{tabular}
	\end{center}
\end{table}

\newpage
\noindent\textbf{Внешние ключи}

\begin{itemize}
\item В Таблице Users поле user\_role ссылается на поле id таблицы Roles;
\item В Таблице Incidents 
	\begin{itemize}
		\item поле incident\_type ссылается на поле id таблицы Types;
		\item поле incident\_status ссылается на поле id таблицы Statuses;
		\item поле author ссылается на поле id таблицы Users.
	\end{itemize}
\end{itemize}

\noindent\textbf{Ролевая модель}

В базе данных существуют четыре роли:
\begin{enumerate}
\item Администратор -- имеет доступ к всем таблицам, доступны все операции над ними;
\item Модератор -- имеет доступ к таблицам Incidents, Users; над таблицами доступны операции: SELECT, INSERT, UPDATE, DELETE;
\item Авторизированный пользователь -- доступ к таблице Incidents; над таблицами доступны операции: SELECT, INSERT; 
\item Гость -- доступны операция SELECT над таблицей Incidents и операция INSERT над таблицей Users.
\end{enumerate}

\noindent\textbf{Триггер}

Для функционирования системы необходимо постоянное наличие в ней хотя бы одного модератора. Необходим механизм, который будет препятствовать нарушению этого условия.

Для решения этой задачи можно использовать триггер, который будет срабатывать при удалении пользователя или обновлении его роли. Если операция производится с единственным модератором в системе, ее выполнение будет остановлено. 

Ниже представлена схема алгоритма функции, выполняемой триггером.

\imgw{trigger}{h!}{0.8\textwidth}{Схема алгоритма функции триггера check\_moderators()}

\newpage
\section*{Вывод}
В данном разделе:
\begin{itemize}
\item спроектированы сущности базы данных;
\item описаны связи сущностей с помощью внешних ключей;
\item описана ролевая модель для разграничения доступа к базе данных;
\item описан триггер для корректного функционирования системы.
\end{itemize}