\chapter{Технологическая часть}
В данном разделе будет обоснован выбор СУБД, описаны средства реализации и интерфейс ПО.  

\section{Выбор СУБД}
Для хранения данных приложения выбрана свободная объектно-реляционная система управления базами данных PostgreSQL. Данная СУБД поддерживает все необходимые для работы приложения типы данных. Она отвечает требованиям ACID, а также поддерживает целостность данных за счет наличия таких средств, как первичные и внешние ключи, ограничения NOT NULL, прочие проверочные ограничения. 

\section{Выбор средств реализации}
Приложение для работы с базой данных, представляет из себя Web-сервер, доступ к которому осуществляется с помощью REST API \cite{rest-api}.  

Для реализации сервера используется язык программирования Go \cite{golang}. Это компилируемый многопоточный, статически типизированный язык программирования, подходящий для создания простых, но эффективных Web-сервисов.  

Для взаимодействия с базой данных используется пакет github.com/jmoiron/sqlx\cite{sqlx}, расширяющий стандартный пакет языка database/sql\cite{gosql}, а также драйвер github.com/lib/pq\cite{pq} для работы с СУБД PostgreSQL.  

Для реализации REST API используется Web фреймворк Gin \cite{gin}. 

Для реализации пользовательского Web-интерфейса использованы язык гипертекстовой разметки документов HTML\cite{html} и язык программирования JavaScript\cite{js}. Для отображения экологических инцидентов на интерактивной карте использован API Яндекс.Карт\cite{map-api}.
 
\section{Детали реализации}

\noindent\textbf{Создание и наполнение базы данных}

SQL-скрипты создания базы данных, создания таблиц, установки ограничений, создания ролевой модели, создания триггера представлены в листингах ~\ref{lst:sql-1}~---~\ref{lst:sql-2}. Наполнение базы данных происходило с помощью заготовленных .csv файлов. 

\noindent\textbf{Паттерн взаимодействия с данными}

Для взаимодействия с данными используется паттерн "Репозиторий". Он позволяет отделить логику приложения от деталей реализации слоя доступа к данным. Структура, реализующая репозиторий для сущности Incidents базы данных, приведена в листингах ~\ref{lst:repo-1}~---~\ref{lst:repo-2}.

\noindent\textbf{REST API}

\texttt{REST API} интерфейс приложения представлен в Таблице~\ref{tbl:rest-api}.

%\begin{landscape}
\begin{longtable}{|p{0.3\textwidth}|p{0.125\textwidth}|p{0.5\textwidth}|} 
    \caption{Описание реализованного \texttt{REST API}}\label{tbl:rest-api}\\\hline
        Путь & Метод & Описание \\
    \endfirsthead

    \multicolumn{3}{l}
    {{\tablename\ \thetable{} -- продолжение}} \\\hline 
        Путь & Метод & Описание \\
    \endhead
    
    \multicolumn{3}{|r|}{{Продолжение на следующей странице}} \\ \hline
    \endfoot
    
    \multicolumn{3}{|r|}{{Конец таблицы}} \\ \hline
    \endlastfoot
     \hline 
     
    /api/auth/sign-up & POST & Регистрация пользователя \\\hline
    /api/auth/sign-in & POST & Вход в систему \\\hline
    /api/users/ & GET & Получение списка всех пользователей \\\hline
    /api/users/{id} & GET & Получение пользователя с заданным id \\\hline
    /api/users/{id} & PUT & Обновление данных пользователя с заданным id \\\hline
    /api/users/{id} & DELETE & Удаление пользователя с заданным id \\\hline
    /api/users/{id}/role & PUT & Изменение роли пользователя с заданным id \\\hline
    /api/incidents/ & GET & Получение списка всех инцидентов \\\hline
    /api/incidents/ & POST & Создание инцидента \\\hline
    /api/incidents/{id} & GET & Получение инцидента с заданным id \\\hline
    /api/incidents/{id}] & PUT & Обновление данных об инциденте с заданным id \\\hline
    /api/incidents/{id} & DELETE & Удаление инцидента с заданным id \\\hline
    /api/incidents/type/{type} & GET & Получение списка инцидентов с заданным типом \\\hline
    /api/roles/ & GET & Получение списка всех ролей \\\hline
    /api/roles/ & POST & Создание роли \\\hline
    /api/roles/{id} & GET & Получение роли с заданным id \\\hline
    /api/roles/{id}] & PUT & Обновление данных о роли с заданным id \\\hline
    /api/roles/{id} & DELETE & Удаление роли с заданным id \\\hline
    /api/statuses/ & GET & Получение списка всех статусов инцидентов \\\hline
    /api/statuses/ & POST & Создание статуса \\\hline
    /api/statuses/{id} & GET & Получение статуса с заданным id \\\hline
    /api/statuses/{id}] & PUT & Обновление данных о статусе с заданным id \\\hline
    /api/statuses/{id} & DELETE & Удаление статуса с заданным id \\\hline
    /api/types/ & GET & Получение списка всех типов инцидентов \\\hline
    /api/types/ & POST & Создание типа \\\hline
    /api/types/{id} & GET & Получение типа с заданным id \\\hline
    /api/types/{id}] & PUT & Обновление данных о типе с заданным id \\\hline
    /api/types/{id} & DELETE & Удаление типа с заданным id \\\hline   
\end{longtable}
%\end{landscape}

\noindent\textbf{Web-интерфейс}

Web-интерфейс предназначен для просмотра информации об экологических инцидентах в виде списка, а также в виде отметок на интерактивной карте. Демонстрация Web-интерфейса представлена на рисунке \ref{img:web}.

\section*{Вывод}
В данном разделе:
\begin{itemize}
	\item определена СУБД для решения задачи;
	\item определены средства реализации ПО: используемые языки программирования и библиотеки;
	\item приведены детали реализации ПО: SQL-скрипты, слой взаимодействия с данными, REST API интерфейс, Web-интерфейс.
\end{itemize}

