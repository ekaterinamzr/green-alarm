\chapter*{ЗАКЛЮЧЕНИЕ}
\addcontentsline{toc}{chapter}{ЗАКЛЮЧЕНИЕ} 

В результате выполнения курсовой работы была разработана база данных для хранения данных об экологических инцидентах.

Для достижения данной цели были решены следующие задачи:
\begin{itemize}
	\item формализована задача и сформулированы требования к разрабатываемому ПО;
	\item проанализированы существующие СУБД и выбрана подходящую для решения задачи систему;
	\item спроектирована база данных, описаны ее сущности и связи;
	\item реализован интерфейс доступа к базе данных;
	\item реализовано ПО для работы пользователей с базой данных.
\end{itemize}  

В ходе выполнения исследовательской части работы было установлено, что выполнение операци вставки, обновления и удаления выполняются примерно в 1.5 раза быстрее при использовании документоориентированной СУБД MongoDB. Поиск по ID выполняется примерно за равное время независимо от использованной СУБД (MongoDB быстрее в 1.1 раза). Поиск списка инцидентов (по типу) выполняется в 2.1 раза быстрее при использовании реляционной СУБД PostgreSQL.  
 
