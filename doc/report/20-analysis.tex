\chapter{Аналитическая часть}
В данном разделе будут рассмотрены существующие решения, формализована решаемая задача, выбран способ хнанения данных. 

\section{Обзор существующих решений}
В 2020 году WWF России представил проект, который призван помочь жителям нашей страны оперативно сообщать об авариях и инцидентах \cite{wwf1}. Идея была размещена в рамках форума "Сильные идеи для нового времени" \cite{forum}. С помощью специальной формы на сайте каждый сможет оперативно передать фото с места аварии, координаты инцидента и снабдить свое сообщение комментарием. По информации с места событий будут запрашиваться оперативные данные космического мониторинга. Однако на момент написания курсовой работы нет сведений о судьбе данного проекта.

WWF России и Fairy, бренд компании Procter\&Gamble, запустили национальную программу общественного мониторинга аварийных экологических ситуаций. Программа позволяет своевременно отслеживать и оповещать дежурные службы об инцидентах, аналогичных катастрофам на побережье Авачинской бухты на Камчатке, разливе в Норильске или аварии на продуктопроводе на реке Оби \cite{wwf2}. В рамках данного проекта у пользователей есть возможность получить информацию об инцидентах в с помощью интерактивной карты. Однако пользователи не могут загружать информацию об инцидентах в систему. Также данная программа узконаправлена -- данные собираются только об инцидентах в нефтегазовом секторе. 

\section{Формализация задачи}
Под экологическим инцидентом будем понимать любое происшествие, которое привело или может привести к неблагоприятным последствиям для окружающей среды. 

Разрабатываемая система должна выделять следующие типы экологических инцидентов: 
\begin{itemize}
	\item разлив нефти или нефтепродуктов;
	\item выброс радиоактивных веществ;
	\item выброс аварийно химически опасных веществ;
	\item выброс биологически опасных веществ;
	\item пожар;
	\item несанкционированная свалка, скопление мусора;
	\item другие экологические инциденты.
\end{itemize}

Каждая запись об экологическом инциденте должна содержать следующие данные:
\begin{itemize}
	\item название (краткое описание);
	\item тип;
	\item координаты;
	\item дата;
	\item статус (подтвержден/не подтвержден);
	\item пользователь, опубликовавший инцидент.
\end{itemize}

Каждая запись об экологическом инциденте может содержать следующие данные:
\begin{itemize}
	\item комментарий.
\end{itemize}

Работа пользователей с базой экологических инцидентов должна осуществляться посредством клиент-серверного веб-приложения с возможностью авторизации. Приложение должно поддерживать работу четырех типов пользователей со следующими возможностями: 
\begin{itemize}
	\item неавторизованный пользователь:
	\begin{itemize}
		\item просмотр записей об экологических инцидентах в виде списка;
		\item просмотр записей об экологических инцидентах в виде карты;
		\item регистрация нового аккаунта или вход в существующий;
	\end{itemize}
	\item авторизованный пользователь:
	\begin{itemize}
		\item все возможности неавторизованного пользователя;
		\item добавление записи об экологическом инциденте;
		\item выход из аккаунта;
	\end{itemize}
	\item модератор:
	\begin{itemize}
		\item все возможности авторизованного пользователя;
		\item подтвеждение инцидента (установка соответствующего статуса);
		\item удаление записи об инциденте;
		\item редактирование записи об инциденте;
	\end{itemize}
	\item администратор:
	\begin{itemize}
		\item все возможности модератора;
		\item назначение авторизованному пользователю роли модератора;
		\item снятие авторизованного пользователя с роли модератора;
		\item редактирование существующих в системе типов инцидентов;
		\item редактирование существующих в системе статусов инцидентов; 
		\item редактирование существующих в системе ролей пользователей. 
	\end{itemize}
\end{itemize}

Ниже представлена диаграмма использования приложения.

\imgw{usecase}{h!}{1.0\textwidth}{Диаграмма использования приложения}

\clearpage

\section{Модели баз данных}

Модель базы данных -- это тип модели данных, которая определяет логическую структуру базы данных и в корне определяет, каким образом данные могут храниться, организовываться и обрабатываться \cite{db-model}.
Рассмотрим основные модели.  

\noindent\textbf{Иерархическая модель}

В иерархической модели информация организована в виде древовидной структуры.  
Каждая запись имеет одного "родителя" и несколько потомков. 

Такие модели данных графически могут быть представлены в виде перевернутого дерева.  
Оно состоит состоящее из объектов, каждый из которых имеет уровень. Корень дерево -- первый уровень, его потомки второй и так далее.

Основной недостаток иерархической модели данных -- невозможно реализовать отношение "many-to-many", связь, при которой у потомка существует несколько родителей.  

В качестве примера такой модели можно привести каталог в операционной системе Windows.  

\noindent\textbf{Сетевая модель}

Сетевая модель -- это структура, у которой любой элемент может быть связан с любым другим элементом.  

Сетевая модель описывается как иерархическая, но в отличие от последней лишена недостатков, связанных с невозможностью реализовать "many-to-many" связь.  
Разница между сетевой и иерархической заключается в том, что в сетевой модели у потомка может быть несколько предков, когда у иерархической только один.  

Основной недостаток -- жёсткость задаваемых структур и сложность изменения схем БД из-за реализации связей между объектами на базе физических ссылок (через указатели на объекты).  

В качестве примера такой модели можно привести WWW (World Wide Web).  

\noindent\textbf{Реляционная модель}

Данные в реляционной модели хранятся в виде таблиц и строк, таблицы могут иметь связи с другими таблицами через внешние ключи, таким образом образуя некие отношения.

Реляционные базы данных используют язык SQL. Структура таких баз данных позволяет связывать информацию из разных таблиц с помощью внешних ключей (или индексов), которые используются для уникальной идентификации любого атомарного фрагмента данных в этой таблице. Другие таблицы могут ссылаться на этот внешний ключ, чтобы создать связь между частями данных и частью, на которую указывает внешний ключ.

SQL используют универсальный язык структурированных запросов для определения и обработки данных. Это накладывает определенные ограничения: прежде чем начать обработку, данные надо разместить внутри таблиц и описать.

\noindent\textbf{Нереляционная модель}

Данные нереляционных баз данных не имеют общего формата. Они могут представляться в виде документов (MongoDB \cite{mongodb}, Tarantool \cite{tarantool}), пар ключ-значение (Redis \cite{redis}), графовых представляниях.

Динамические схемы для неструктурированных данных позволяют:
\begin{itemize}
    \item ориентировать информацию на столбцы или документы;
    \item основывать ее на графике;
    \item организовывать в виде хранилища Key-Value;
    \item создавать документы без предварительного определения их структуры, использовать разный синтаксис;
    \item добавлять поля непосредственно в процессе обработки.
\end{itemize}

\subsection{Выбор модели базы данных}

Для решения задачи будет использоваться реляционная модель данных по нескольким причинам:

\begin{itemize}
    \item изложение данных будет осуществляться в виде таблиц;
    \item данные структурированные, структура нечасто измяема;
    \item возможность исключить дублирование, используя связь между отношениями с помощью внешних ключей;
    \item разделение доступа к данным от способа их физической организации.
\end{itemize} 

\newpage
\section{Формализация данных}

Ниже представлена ER-диаграмма сущностей в нотации Чена.
\imgw{entities}{h!}{1.0\textwidth}{ER-диаграмма сущностей в нотации Чена}

Хранение данных пользователей и информации об инцидентах должно происходить в одной базе данных.
Пользователи и инциденты должны иметь уникальные идентификаторы, чтобы их можно было однозначно идентифицировать.

Пользователи и инциденты связаны отношением один ко многим, информация об этих связях должна сохраняться.

Роли пользователей, типы инцидентов, статусы инцидентов должны храниться в виде пар: уникальный идентификатор -- имя. 


\section*{Вывод}

В данном разделе:
\begin{itemize}
    \item рассмотрены существующие решения;
    \item формализована решаемая задача;
    \item формализированы данные, используемые при решении задачи;
    \item выбрана модель баз данных для решения задачи.
\end{itemize}
