\chapter{Исследовательская часть}
В данном разделе будет проведен эксперимент по сравнению производительности реляционной и документоориентированной СУБД.

\section{Цель исследования}

Цель исследования -- сравнить время, которое требуется для выполнения операций вставки, удаления, обновления и получения данных с помощью реляционной СУБД PostgreSQL и документоориентированной СУБД MongoDB.  

Для достижения цели требуется:
\begin{itemize}
\item создать базу данных MongoDB для хранения сущностей экологических инцидентов;
\item создать слой доступа к данным (паттерн "Репозиторий") MongoDB;
\item сравнить скорость выполнения операций вставки, удаления, обновления и получения данных при использовании репозиториев PostgreSQL и MongoDB.
\end{itemize}

\section{Описание исследования}
Для выполнения исследования была создана база данных MongoDB с коллекцией экологических инцидентов. Далее в приложении был реализован слой доступа к данным из базы данных MongoDB (паттерн "Репозиторий"), который представлен в листингах \ref{lst:mongorepo-1}--\ref{lst:mongorepo-4}.

В ходе эксперимента выполнялись замеры времени (мс) для следующих операции с каждой из баз данных с помощью соответствующих репозиториев: 
\begin{itemize}
	\item 1000 операций вставки;
	\item 1000 операций обновления;
	\item 1000 операций поиска по id (один результат);
	\item 1000 операций поиска по типу (список результатов);
	\item 1000 операций удаления.
\end{itemize}

Код проведения исследования представлен в листингах \ref{lst:research-1}--\ref{lst:research-4}. 

\newpage

\section{Технические характеристики}
Ниже приведены технические характеристики устройства, на котором было проведенно тестирование ПО:

\begin{itemize}
	\item Операционная система: Ubuntu 20.10 \cite{ubuntu} Linux \cite{linux} 64-bit;
	\item Оперативная память: 12 GB;
	\item Процессор: AMD® Ryzen 7 3700u with radeon vega mobile gfx × 8
	\cite{amd}.
\end{itemize}

\section{Результаты исследования}
Конечный результат сформирован усредненными значениями времени (мс), полученными по результатам 100 экспериментов и представлен в виде таблицы: 

\begin{table}[h]
	\begin{center}
		\caption{\label{tbl:research} Результаты исследования (время выполнения 1000 операций в мс)}
		\begin{tabular}{|p{24mm}|p{24mm}|p{24mm}|p{24mm}|p{24mm}|p{24mm}|}
			\hline
			СУБД & Вставка & Обновление & Удаление & Поиск одного & Поиск списка \\ \hline
			PostgreSQL & 731 & 705 & 649 & 417 & 1932\\ \hline
			MongoDB & 516 & 477 & 430 & 472 & 4158\\ \hline
		\end{tabular}
	\end{center}
\end{table}

\section*{Вывод}

В ходе исследования в приложении была реализована поддержка документоориентированной СУБД MondoDB, создана база данных, написан слой доступа к ее данным. Были проведены замеры времени, которое требуется для выполнения операций вставки, удаления, обновления и получения данных с помощью PostgreSQL и MongoDB.

Исследование показало, что выполнение операци вставки, обновления и удаления выполняются примерно в 1.5 раза быстрее при использовании MongoDB. Поиск по ID выполняется примерно за равное время независимо от использованной СУБД (MongoDB быстрее в 1.1 раза). Поиск списка инцидентов (по типу) выполняется в 2.1 раза быстрее при использовании PostgreSQL. 

