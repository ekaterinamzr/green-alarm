\chapter*{Введение}
\addcontentsline{toc}{chapter}{Введение}
%(TODO: ссылки на катастрофы, цитату https://www.forbes.ru/forbeslife/410857-my-ostaemsya-v-plenu-sovetskih-tehnologiy-i-okazhemsya-v-glubokom-ekonomicheskom)
Разлив дизельного топлива в Норильске\cite{diesel}, сибирские пожары\cite{fires}, катастрофа на Камчатке\cite{kamchatka}, повлекшая за собой массовую гибель морских животных -- последние годы подобные инциденты потрясают экологов и всех небезразличных жителей планеты. Чаще всего первоисточником информации о происшествиях являются местные жители, которые, обнаружив проблему, начинают звонить журналистам, писать экологическим организациям, публиковать информацию в социальных сетях\cite{interview}.  

Необходим единый сервис для размещения информации об экологических инцидентах. Такая система не только позволила бы оперативно оповещать специальные службы и экологичсекое сообщество о происшествиях, но и повысила бы экологическую осведомленность населения. Сформированная таким образом база данных экологических инцидентов могла бы использоваться исследователями для анализа.  

Цель данной работы -- разработать базу данных для хранения данных об экологических инцидентах.

Для достижения данной цели, необходимо решить следующие задачи:
\begin{itemize}
	\item формализовать задачу и сформулизовать требования к разрабатываемому ПО;
	\item проанализировать существующие СУБД и выбрать подходящую для решения задачи систему;
	\item спроектировать базу данных, описать ее сущности и связи;
	\item реализовать интерфейс доступа к базе данных;
	\item реализовать ПО для работы пользователей с базой данных.
\end{itemize}  